

\chapter{الدراسة المرجعية}
يبيّن هذا الفصل الدراسة المرجعية للمشروع.
يبدأ بتقديم مفاهيم تعلّم الآلة ومراحلها المختلفة والمعايير المعتمدة لتقييمها.
ويقدّم مفاهيم ومراحل معالجة اللغات الطبيعية.
وأخيراً يسرد بعض الأوراق الأبحاث العلمية المتعلقة بهذا المشروع، ويوضح المنهجيات المتبعة فيها.

\section{تعلم الآلة}
تعلم الآلة \LR{Machine Learning} هو فرع جزئي من الذكاء الصنعي \LR{Artificial Intelligence}.
يُقصد بتعلم الآلة مجموعة الأدوات والمفاهيم والمنهجيات المستخدمة لبرمجة الحواسيب
بطريقة تسمح لهذه الحواسيب بالتعلم من المعطيات~\cite{hands-on}.

ويمكن أيضاً تعريفه بشكل أكثر عمومية كالتالي:

\begin{english}
	``Machine Learning is the field of study that gives computers the ability to learn
	without being explicitly programmed.'' \\
	--Arthur Samuel, 1959
\end{english}

كما يعتبر التعريف التالي تقني وأكثر دقة:

\begin{english}
	``A computer program is said to learn from experience $E$ with respect to some task $T$
	and some performance measure $P$, if its performance on $T$, as measured by $P$, improves
	with experience $E$.'' \\
	--Tom Mitchell, 1997
\end{english}

على سبيل المثال، النظام الذي يقوم بفلترة الإيميلات إلى إيميلات مؤذية \LR{spam} وإيميلات غير مؤذية \LR{non-spam}، يستخدم منهجيات تعلم الآلة.
يقوم هذا النظام بتعلم طريقة التمييز بين هذين النوعين من الإيميلات باستخدام عدد كبير من الأمثلة والمعطيات المصنفة مسبقاً.
نسمي هذه المجموعة من الأمثلة بمعطيات التدريب \LR{Training Set}، وكل مثال منها نسميه مثال تدريبي \LR{Training Instance}.

في هذه الحالة، المهمة $T$ هي تصنيف الإيميلات الجديدة إلى إيميلات مؤذية وإيميلات غير مؤذية، الخبرة $E$ هي مجموعة معطيات التدريب،
ومؤشر قياس الأداء $P$ يمكن تعريفه بعدّة طرق؛ فمثلاً يمكننا استخدام نسبة نسبة عدد الإيميلات التي تم تصنيفها بشكل صحيح إلى عدد الإيميلات الكلي
(هذا المعيار يسمى الدقّة \LR{Accuracy} كم سنرى لاحقاً).

\subsection{تصنيفات تعلم الآلة}
يمكن تصنيف أنظمة تعلم الآلة وفق عدّة معايير. التصنيف الأكثر شهرة يعتمد على آلية التدريب، وهو كالتالي:
\begin{itemize}
	\item
	التعلم تحت الإشراف \LR{Supervised Learning}:
	وهي حالة أن تكون الأمثلة التدريبية متوفرة مع الخرج \LR{label} المرتبط بها. وهذه حالة مثال تصنيف الإيميلات المطروح سابقاً.
	حيث أن معطيات التدريب هي مجموعة كبيرة من الإيميلات المصنفة مسبقاً من قبل البشر إلى إيميلات مؤذية وإيميلات غير مؤذية.
	\item
	التعلم بدون إشراف \LR{Unsupervised Learning}:
	وهي حالة أن تكون معطيات التدريب موجودة ولكنها غير مصنفة \LR{unlabeled} أو غير مرتبطة بخرج معيّن.
	على سبيل المثال، قد ترغب شركة في تصنيف زبائنها إلى عدّة مستويات، زبائن من الدرجة الأولى، زبائن من الدرجة الثانية، وهكذا.
	فيمكن استخدام تعلم الآلة لاكتشاف بعض الأنماط الموجودة في معطيات الزبائن واكتشاف هكذا تصنيف.
	وهذا ما يُعرف بالتجميع \LR{Clustering}.
	\item
	التعلم نصف المشرف عليه \LR{Semi-Supervised Learning}:
	وهي حالة وسيطة بين التصنيفين السابقين. تكون فيها بعض أمثلة التدريب مرتبطة بخرج معيّن (غالباً تشكل النسبة الصغيرة)،
	وتكون باقي الأمثلة غير مرتبطة بخرج. تنطبق هذه الحالة على مثال تصنيف الإيميلات في حال لم تكن جميع معطيات التدريب مصنفة بشكل مسبق.
	\item 
	التعلم بالتعزيز \LR{Reinforcement Learning}:
	وهي الحالة التي يتخاطب فيها النظام مع بيئة أخرى. تقدم له هذه البيئة نتائج \LR{feedback} بناءً على أفعاله.
	هذا الصنف ينطبق على الخوارزميات المستخدمة لتدريب الأنظمة التي تتعلم الألعاب.
	حيث يقوم النظام بمجموعة من الأفعال \LR{actions} ضمن بيئة اللعبة،
	وبناءً على النتائج (تحسّن نتيجته أو انخفاضها) يغيّر أفعاله اللاحقة.
\end{itemize}

وعلى وجه الخصوص يمكن تصنيف التعلم تحت الإشراف بحسب نوع الخرج المرتبط بمعطيات التدريب.
تصنّف بشكل أساسي عريض كالتالي:
\begin{itemize}
	\item 
	التصنيف \LR{Classification}:
	يكون الخرج المرتبط بكل مثال تدريبي هو صف \LR{class} محدد من مجموعة صفوف. عدد هذه الصفوف قد يكون 2، 3، إلخ.
	في مثال تصنيف الإيميلات السابق، عدد الصفوف هو 2، حيث أن كل مثال تدريبي (إيميل معيّن من معطيات التدريب) هو إمّا مؤذي أو غير مؤذي.
	\item 
	الانحدار \LR{Regression}:
	يكون الخرج المرتبط بكل مثال تدريبي هو عدد حقيقي. مثل مسألة التنبؤ بسعر منزل بمعرفة معلومات عنه مثل مساحته، عدد الغرف، إلخ.
\end{itemize}

\subsection{خوارزميات تعلم الآلة}



