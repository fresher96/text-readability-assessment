

\section*{الملخص}
\addcontentsline{toc}{section}{الملخص}
تُعد القراءة نشاط مهم نمارسه في حياتنا اليومية،
سواء للمطالعة أو قراءة الأخبار أو التعلم أو غيرها.
فيهتم هذا المشروع ببناء مُصنّف قادر على تقييم مقروئية النصوص المكتوبة باللغة الانكليزية.
وتصنيفها وفق عدّة مستويات (سهل، متوسط، صعب).
علماً بأن عدد هذه المستويات والتباين بين صعوبتها يتعلق بالمعطيات التي تم استخدامها لبناء هذا المصنّف.
تتمحور منهجية العمل حول استخراج مجموعة من الميزات من هذه النصوص.
ميزات تقليدية (مثل طول النص)، ومفرداتية (تعبّر عن تنوع المفردات المستخدمة)، ونحوية (تعكس الصياغة والتراكيب المستخدمة).
ثمّ استخدام خوارزميات تعلم الآلة (تم استخدام الـ \eng{SVM} بشكل أساسي) لتدريب وبناء مصنّف باستخدام هذه الميزات.
تم استخدام مجموعة المعطيات \eng{One Stop English Corpus} والحصول على نسبة صِحّة $80.83\%$.
علماً أن أعلى نسبة صِحّة تم الحصول عليها باستخدام هذه المعطيات هي $78.13\%$.
وأيضاً خلال المشروع تم بناء مكتبة بلغة جافا لاستخراج الميزات من النصوص.
فيمكن بسهولة استخدامها لاستخراج الميزات التي تم تنجيزها من نص ما، أو توسيع هذه المكتبة بتعريف ميزات جديدة.


\vfill
\selectlanguage{english}
\section*{Abstract}

Reading is an important activity in our daily lives.
We read the news, we read to learn, etc.
This project aims to build a classifier to automatically assess text readability for texts written in English.
By classifying them into different levels (easy, medium, and hard).
Noting that, the number of levels and the variance of their difficulty depends on the data used to build the classifier.
The methodology is to extract certain set of features from these texts.
Traditional features (e.g. text length), lexical features (measure vocabulary being used),
and syntactic features (measure sentence complexity).
Then using machine learning algorithms (SVM mainly) to train and build a classifier using these features.
Using One Stop English Corpus as the dataset, we achieved an accuracy of $80.83\%$.
Given that the best achieved accuracy on this dataset is $70.13\%$.
Also, we implemented a library in java to extract features from texts.
It can be easily used to extract the features we implemented from a given text.
It also can be extended by implementing new features.

\selectlanguage{arabic}





