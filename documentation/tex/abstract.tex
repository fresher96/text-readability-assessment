

\section*{الملخص}
\addcontentsline{toc}{section}{الملخص}
تُعد القراءة نشاطاً مهماً نمارسه في حياتنا اليومية،
سواء لغرض المطالعة أو قراءة الأخبار أو التعلم أو غيرها.
يهدف هذا المشروع إلى بناء مُصنّفٍ قادرٍ على تقييم مقروئية النصوص المكتوبة باللغة الانكليزية،
وتصنيفها وفق مستويات متدرجة (سهل، متوسط، صعب).
يختلف عدد هذه المستويات ودرجة التباين بين صعوبتها تبعاً للمعطيات التي يتم استخدامها لبناء المصنّف.
تتمحور منهجية العمل حول استخراج مجموعة من السمات من هذه النصوص:
سمات تقليدية (مثل طول النص)، ومفرداتية (تعبّر عن تنوع المفردات المستخدمة)، ونحوية (تعكس الصياغة والتراكيب المستخدمة).
بعد ذلك يجري استخدام خوارزميات تعلم الآلة (تم استخدام الـ \eng{SVM} بشكل أساسي) لتدريب وبناء مصنّف باستخدام هذه السمات.
جرى استخدام مجموعة المعطيات \eng{One Stop English Corpus}، وكانت نسبة الصِحّة المحققة $80.83\%$ وذلك باستخدام $30$ سمة.
علماً أن أعلى نسبة صِحّة في الأدبيات، تم الحصول عليها باستخدام هذه المعطيات هي $78.13\%$ وذلك باستخدام $155$ سمة~\cite{vajjala2018}.
كما جرى بناء مكتبة بلغة جافا لاستخراج السمات من النصوص،
يمكن بسهولة استخدامها لاستخراج السمات التي تم تنجيزها من نص ما، أو توسيع هذه المكتبة بتعريف سمات جديدة.


\vfill
\selectlanguage{english}
\section*{Abstract}

Reading is an important activity in our daily lives.
We read the news, we read to learn, etc.
This project aims to build a classifier to automatically assess text readability for texts written in English.
By classifying them into different levels (easy, medium, and hard).
Noting that, the number of levels and the variance of their difficulties depend on the data used to build the classifier.
The methodology is to extract certain set of features from these texts.
Traditional features (e.g. text length), lexical features (measure vocabulary being used),
and syntactic features (measure sentence complexity).
Then using machine learning algorithms (SVM mainly) to train and build a classifier using these features.
Using One Stop English Corpus as the dataset, we achieved an accuracy of $80.83\%$ (using $30$ feature).
Given that the best achieved accuracy on this dataset is $78.13\%$ using $155$ feature~\cite{vajjala2018}.
Also, we implemented a library in java to extract features from texts.
It can be easily used to extract the features we implemented from a given text.
It also can be extended by implementing new features.

\selectlanguage{arabic}





