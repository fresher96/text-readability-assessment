

\chapter{خاتمة المشروع}
يختتم هذا الفصل المشروع.
فيبيّن الفائدة المكتسبة من المشروع، والمشكلات والصعوبات التي واجهتنا خلاله.
ويبيّن بعض الآفاق المستقبلية.

\section{الخاتمة والفائدة المكتسبة}
في هذا المشروع، تم بناء تطبيق لتقييم مقروئية نصوص اللغة الانكليزية.
بالإضافة إلى تنجيز مكتبة سهلة الاستخدام وقابلة للتوسع لاستخراج الميزات من النصوص.
تم التعرف على عدد واسع من المفاهيم والطرائق خلال تنفيذ هذا المشروع.
فتم التعرف على مفاهيم تعلم الآلة المختلفة لأول مرّة،
بدءاً من مرحلة تنظيف المعطيات، وحتى مرحلة استخدام النموذج الناتج في تطبيق نهائي.
والتعرف على عدد من الأدوات المفيدة لتطبيقها بشكل عملي.
كما تم التعرف على مفاهيم معالجة اللغات الطبيعية ومختلف الأدوات المتوفرة لها.
وتم بناء كود برمجي ليس بحجم صغير مما أدى إلى اكتساب خبرات في تصميم المكونات البرمجية والتعرف على أنماط تصميمية جديدة.



\section{المشكلات والصعوبات}
الصعوبات الأساسية التي واجهتنا في المشروع تتمثل بـ:
\begin{itemize}
	\item 
	الحاجة إلى دراسة مرجعية مطوّلة كوننا نتعامل مع مفاهيم تعلم الآلة لأول مرة.
	\item 
	تنوّع المهام ضمن المشروع.
	فتوجد عدّة مراحل ضمن المشروع، كل منها يحتاج وقت وجهد.
	مثل الحاجة إلى تجهيز المعطيات وتنظيفها.
	والحاجة إلى كتابة كود برمجي كبير نسبياً لاستخراج مجموعة واسعة من الميزات التي تختلف طرائق حسابها.
	والحاجة إلى دراسة عدة أدوات ومعرفة آلية استخدامها.
	\item
	صعوبة الموضوع المدروس.
	إن عملية تقييم مقروئية نص ليست بأمر سهل. ومعايير التقييم ليست دقيقة.
	فقد يتدخل فيها طابع شخصي؛ أي يختلف التقييم باختلاف الشخص.
	وقد يسبب ذلك عدم دقّة في المعطيات التي تم استخدامها للتدريب.
\end{itemize}


\section{آفاق مستقبلية}
توجد العديد من الآفاق المستقبلية التي يمكن طرحها بناءً على هذا المشروع.
فيمكن تطوير هذا المشروع بتمديده إلى لغات أخرى.
ويمكن توسيع التطبيق النهائي بإضافة مُصنفات أخرى لتوسيع قابلية استخدامه.
ويمكن إضافة مجموعة جديدة من الميزات وتحسين أداء عملية التصنيف.
ويمكن تفصيل التقييم الكلي إلى عدّة جوانب، مثل إظهار تقييم للمفردات، وإظهار تقييم للصياغة، وهكذا.

أيضاً بالإمكان استخدام هذا المشروع كمكوّن جزئي ضمن نظام أوسع.
فيمكن إضافته ضمن محرر نصوص أو ضمن المتصفح،
بحيث يمكن للمستخدم تحديد نص وتقييم مقروئيته.
مما سيزيد من قدرات محررات النصوص الحالية، ويجعل عملية الكتابة أسهل.
ومن الممكن إضافته ضمن محركات البحث بطريقة تسمح للمستخدم بالبحث عن مقال معين وبصعوبة معينة.






