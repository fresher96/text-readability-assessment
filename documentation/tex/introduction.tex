

\chapter{التعريف بالمشروع}
يُمهّد هذا الفصل للمشروع، حيث يُبيّن فكرة المشروع وأهميتها والأهداف المرجوّة منه. ويذكر المتطلبات الوظيفية وغير الوظيفية للمشروع.


\section{مقدمة}
تلعب القراءة دور مهم جداً في تعلم لغة جديدة أو لاكتساب معارف ومعلومات حول موضوع معيّن.
بالتالي فإن أي مسببات للصعوبة أثناء عملية القراءة ستؤثر سلباً في عملية التعلم واكتساب المعارف.
فاهتم الباحثون بالأسباب التي تؤدي إلى صعوبة في قراءة النصوص وتأثيراتها على القُرّاء.
وقد تمت دراسة الخواص اللغوية التي تسبب صعوبة في قراءة النصوص؛
مثل المفردات، والقواعد، والترابط.
وأجريت عدّة دراسات تحاول بناء نماذج لتقييم مقروئية النصوص \eng{Text Readability Assessment}.
إنّ الهدف العريض من هذا المشروع هو بناء تطبيق لتقييم صعوبة قراءة نص معّين بشكل آلي.

\section{أهمية المشروع وتطبيقاته}
إنّ بناء تطبيق يقوم بتقييم صعوبة قراءة نص بشكل آلي هو أداة مفيدة.
فيمكن للاساتذة استخدامه لمساعدتهم في اختيار نصوص مناسبة لطلابهم سواء أثناء الجلسات التعلمية أو في الاختبارات.
خصوصاً اساتذة تعليم اللغات.
كما أنه بوجود معلومات هائلة متاحة على الانترنت،
فإن هذا التطبيق سيساعد الطلاب على اختيار ما يناسبهم أثناء عملية تعلمهم عن موضوع معين أو قراءة مقالات حول مجال ما.
وبعيداً عن سياق الأمور التعليمية،
يمكن لتحليل صعوبة نص أن تكون مناسبة ولازمة في عدّة سيناريوهات مثل تحليل النصوص القانونية والقضائية.
أيضاً يمكن للكُتَّاب الاستفادة من هكذا تطبيق أثناء عملية كتابتهم، سواء كتابة مقال علمي أو مقال صحفي أو خبر أو غيرها.

ولإعطاء تطبيقات ملموسة بشكل أكثر.
سنتحدث لاحقاً عن عدد من النصوص التي تم استخدامها ضمن المشروع،
حيث أن مجموعة اساتذة يختارون نص معيّن ويعيدون صياغته إلى ثلاثة نصوص بما يناسب طلاب من ثلاثة مستويات.
أي أنه ستتم المحافظة على فحوى النص أكثر ما يمكن، ولكن صياغته ستختلف لتناسب ثلاث مستويات من الطلاب.
فوجود هذا التطبيق سيساعدهم في معرفة إذا ما كانت صياغتهم مناسبة أم لا، وهل يحتاجون إلى تبسيطه أكثر من ذلك.

أيضاً يمكن استخدام هذا النص لمساعدة اساتذة اللغة الإنكليزية.
سواء في المعهد العالي أو المدارس أو غيرها.
فعادةً يوجد قسم في امتحان اللغة لتقييم قدرات الطالب على فهم نص جديد في الغة الإنكليزية \eng{reading comprehension}.
إن ما يقوم به الاساتذة أحياناً هو اختيار نص من الكتاب نفسه لم يتم عرضه بشكل مسبق على الطلاب.
أو اختيار نص من الانترنت، وباستخدام هكذا تطبيق تصبح هذه العملية أكثر سهولة ليكون هذا النص أكثر ملائمة لمستوى الطلاب،
وبالتالي أفضل لتقييم الطلاب بشكل سليم وعادل وأكثر موضوعيّة.

\section{المتطلبات}
نسرد فيما يلي المتطلبات الوظيفية والغير وظيفية للمشروع.

\subsection{المتطلبات الوظيفية}

\begin{enumerate}
	\item 
	بناء تطبيق لتقييم سهولة قراءة نص مكتوب باللغة الانكليزية. تحت ما يلي:
	\begin{enumerate}
		\item 
		المصنِّف المستخدم (المستويات التي يتم تصنيف صعوبة النص وفقها، وعددها، والتفاوت بينها)
		يتعلق بالمعطيات المستخدمة للتدريب.
		\item 
		يُتيح التطبيق للمستخدم اختيار واحد من عدّة مُصنفات لتصنيف نص مُدخَل.
		\item 
		تنجيز مُصنّف واحد على الأقل.
	\end{enumerate}

	\item 
	بناء مكتبة برمجية كإطار عمل لاستخراج الميزات لنص أو مجموعة نصوص.
\end{enumerate}




\subsection{المتطلبات الغير وظيفية}
\begin{enumerate}
	\item 
	الفعاليّة والوثوقية. يجب أن يحقق النظام نسبة صحّة مقبولة.
	\item 
	الكفاءة. يستغرق التطبيق وقت بسيط لتصنيف نص معيّن.
	\item 
	يتم تطوير كامل النظام باستخدام لغة البرمجة جافا.
	\item 
	قابلية التوسّع. يمكن إضافة مصنفات جديدة باستخدام المعطيات ذاتها أو باستخدام معطيات جديدة.
	\item 
	يجب أن تحقق مكتبة استخراج الميزات ما يلي:
	\begin{enumerate}
		\item 
		قابلية التوسّع. يمكن لمستخدم المكتبة تنجيز ميزات جديدة.
		\item
		سهولة الاستخدام. يمكن لمستخدم المكتبة استخدام الميزات المنجّزة بشكل مسبق بسهولة والتركيب بينها.
	\end{enumerate}

\end{enumerate}



