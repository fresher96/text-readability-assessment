

\chapter{التعريف بالمشروع}
يُمهّد هذا الفصل للمشروع،
حيث يُبيّن فكرة المشروع وأهميته والأهداف المرجوّة منه.
ويذكر المتطلبات الوظيفية وغير الوظيفية للمشروع.
ويوضح الخطة الزمنية لتنفيذ المشروع.



\section{مقدمة}
تلعب القراءة دوراً مهماً في تعلم لغةٍ جديدةٍ أو اكتساب معارف ومعلومات جديدة حول موضوع معيّن.
بالتالي فإن أي مسببات للصعوبة أثناء عملية القراءة ستؤثر سلباً على عملية التعلم واكتساب المعارف.
اهتم الباحثون بالأسباب التي تؤدي إلى صعوبة قراءة النصوص وتأثيراتها على القُرّاء
حيث تمت دراسة الخصائص اللغوية التي تسبب صعوبة قراءة النصوص؛
سواء على مستوى المفردات أو على مستوى التراكيب والجمل.
وأجريت عدّة دراسات تحاول بناء نماذج لتقييم مقروئية النصوص \eng{Text Readability Assessment}
بشكل آلي، وهو ما سنحاول العمل عليه في هذا المشروع.

\section{أهمية المشروع وتطبيقاته}
تتنوع تطبيقات أنظمة تقييم مقروئية النصوص لتشمل أنواعاً مختلقة من المستخدمين.
فيمكن للاساتذة استخدامه لمساعدتهم في اختيار نصوص مناسبة لطلابهم سواء أثناء الجلسات التعلمية أو في الاختبارات،
وخاصةً أساتذة تعليم اللغات.
هذا التطبيق سيساعد الطلاب على اختيار ما يناسبهم أثناء عملية دراستهم لموضوع معين أو قراءة مقالات حول مجال ما
خاصةً بوجود معلومات هائلة متاحة على الانترنت،.
وبعيداً عن سياق الأمور التعليمية،
يمكن لتحليل صعوبة نص أن تكون مناسبة ولازمة في عدّة سيناريوهات مثل تحليل النصوص القانونية والقضائية.
أيضاً يمكن للكُتَّاب الاستفادة من هكذا تطبيق أثناء عملية كتابتهم، سواء كتابة مقال علمي أو مقال صحفي أو خبر أو غيرها.

ولإعطاء تطبيقات ملموسة بشكل أكثر،
سنتحدث لاحقاً عن عدد من النصوص التي تم استخدامها ضمن المشروع،
حيث يختار مجموعة أساتذة نص معيّن، ويعيدون صياغته إلى ثلاثة نصوص بما يناسب طلاب من ثلاثة مستويات متدرجة.
مع المحافظة على فحوى النص بأكبر شكل ممكن، ولكن صياغته ستختلف لتناسب ثلاث مستويات من الطلاب.
فوجود هذا التطبيق سيساعدهم في معرفة ما إذا كانت صياغتهم مناسبة أم  أنهم يحتاجون إلى تبسيطه بشكل أكبر.

يمكن استخدام هذا التطبيق لمساعدة أساتذة اللغة الإنكليزية.
سواء في المعهد العالي أو المدارس أو غيرها.
فعادةً يوجد قسم في امتحان اللغة لتقييم قدرات الطالب على فهم نص جديد في اللغة الإنكليزية \eng{reading comprehension}.
إن ما يقوم به الاساتذة أحياناً هو اختيار نص من الكتاب نفسه لم يتم عرضه مسبقاً على الطلاب.
أو اختيار نص من الانترنت، فيمكن أن تصبح هذه العملية أكثر سهولة باستخدام هذا التطبيق ليكون هذا النص أكثر ملائمة لمستوى الطلاب،
وبالتالي أفضل لتقييم الطلاب بشكل سليم وعادل وأكثر موضوعيّة.

\section{المتطلبات}
نسرد فيما يلي المتطلبات الوظيفية وغير الوظيفية للمشروع.

\subsection{المتطلبات الوظيفية}

\begin{enumerate}
	\item 
	بناء تطبيق لتقييم سهولة قراءة نص مكتوب باللغة الانكليزية. يتمتع بالمزايا التالية:
	\begin{enumerate}
		\item 
		المصنِّف المستخدم يتعلق بالمعطيات المستخدمة للتدريب.
		(من ناحية المستويات التي يتم تصنيف صعوبة النص وفقها، وعددها، والتفاوت بينها).
		\item 
		يُتيح التطبيق للمستخدم تجريب عدّة مُصنفات لتصنيف نص مُدخَل.
		\item 
		تنجيز مُصنّف واحد على الأقل.
	\end{enumerate}

	\item 
	بناء مكتبة برمجية تساعد على استخراج السمات لنص أو مجموعة نصوص.
\end{enumerate}




\subsection{المتطلبات غير الوظيفية}
\begin{enumerate}
	\item 
	الفعاليّة والوثوقية. يجب أن يحقق النظام نسبة صحّة مقبولة (تتجاوز $75\%$).
	\item 
	الكفاءة. يستغرق التطبيق وقت بسيط لتصنيف نص معيّن (لا يتعدا $10$ ثواني لنص طوله أقل من $1000$ كلمة).
	\item 
	قابلية التوسّع. يمكن إضافة مصنفات جديدة باستخدام المعطيات ذاتها أو باستخدام معطيات جديدة.
	\item 
	يجب أن تحقق مكتبة استخراج السمات ما يلي:
	\begin{enumerate}
		\item 
		قابلية التوسّع. يمكن لمستخدم المكتبة تنجيز سمات جديدة.
		\item
		سهولة الاستخدام. يمكن لمستخدم المكتبة استخدام السمات المنجّزة بشكل مسبق بسهولة والتركيب بينها.
	\end{enumerate}

\end{enumerate}



\section{الخطة الزمنية للمشروع}
لم تكن هناك خطة زمنية واضحة لتنفيذ المشروع.
فقد تم العمل على مراحل جزئية، كل مرحلة استغرقت حوالي أسبوع.
وكان العمل كالتالي:
\begin{enumerate}
	\item 
	الأسبوع الأول 15/7:
	دراسة نظرية (محو أميّة) حول مفاهيم تعلم الآلة وخوارزمياتها المختلفة.
	وتجميع ودراسة بعض الأبحاث النظرية حول تقييم مقروئية النصوص.
	\item
	الأسبوع الثاني 22/7:
	دراسة المنهجيات المستخدمة ضمن الأوراق البحثية حول موضوع المشروع.
	وتأمين مجموعة المعطيات.
	وتحضير قائمة بأهم السمات المستخدمة في هذه الأبحاث.
	\item
	الأسبوع الثالث 22/7:
	تنجيز نموذج أولي \eng{prototype}.
	تنظيف المعطيات.
	تنجيز عملية استخراج السمات من النصوص.
	تطبيق عدد من خوارزميات تعلم الآلة للحصول على نتائج أولية.
	\item
	الأسبوع الرابع 5/8:
	توقف العمل بسبب المشاركة في معسكر تدريبي للتحضير للمسابقة البرمجية السورية \eng{ACM}.
	\item
	الأسبوع الخامس 12/8:
	تحسين وإعادة هيكلة النظام.
	تحسين التنجيز السابق بإعادة تصميم الكود البرمجي.
	وتطبيق خوارزميات جديدة لتحسين النتائج المرحلية.
	\item
	الأسبوع السادس 19/8:
	توقف العمل بسبب عطلة عيد الأضحى.
	\item
	الأسبوع السابع 26/8:
	بناء واجهة التطبيق النهائي.
	البدء بتحضير التقرير النهائي للمشروع.
	\item
	الأسبوع الثامن 2/9:
	استكمال كتابة التقرير.
	تحضير العرض العملي.
	\item
	الأسبوع التاسع 9/9:
	استكمال كتابة التقرير.
	تحضير العرض التقديمي النهائي.
\end{enumerate}








