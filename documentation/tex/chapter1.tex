


\chapter{الاختبارات والنتائج}

\section{أسباب نشوء الدورات الاقتصادية}
\label{sec:cause}
تكمن أسباب نشوء الدورات الاقتصادية في عدم استقرار النشاط الاقتصادي 
سواء في الإنتاج أو في التوزيع؛ وهذا يحدث بسبب كون النظام الاقتصادي معقّد.
وأيضاً من أسباب نشوءها هو عدم تناسق التقدم الاقتصادي
في أكثر القطاعات؛
فبينما يؤدي التقدم التكنولوجي في بعض الصناعات إلى زيادة الإنتاج 
بمعدل أسرع، نجد في الجانب الآخر تجمد النشاط في صناعات أخرى.
ويمكننا أن نضيف أيضاَ أن الطلب على السلع الصناعية%
\footnote{%
	على عكس السلع الاستهلاكية، فهي سلع تستعمل في إنتاج بضائع أخرى أو في تقديم خدمات، وتباع للمستهلك الصناعي.
}
يكون غير منتظم،
فإنّ الطلب على السلع الصناعية ناتج عن الطلب 
على السلع الاستهلاكية. وبما أن أذواق المستهلكين في تغير مستمر
بحيث أنهم يتحولون منم استهلاك سلعة إلى استهلاك سلعة أخرى،
فإن التقلبات في الإنفاق على السلع الاستهلاكية تؤدي إلى تقلبات
أوسع في الطلب على السلع الصناعية.
وأيضاً يمكن القول بأنه من أسباب نشوء الدورات الاقتصادية هو حدوث خلل في
الإنتاج الزراعي بسبب التغيرات في الظروف المناخية والجوية،
والبطء في استجابة العرض للتغيرات في الأسعار وكمية الطلب
مما يؤدي إلى حدوث فجوة بين الطلب و العرض من حين لآخر.

هناك جدل كبير بين الاقتصاديين حول الأسباب الكامنة وراء نشوء الدورات الاقتصادية، مما أدى إلى وضع العديد من النظريات التي تحاول تفسير نشوءها.
أهمها النظريات المناخية، النظريات النقدية، نظريات نقص الاستهلاك، ونظريات الإفراط في الاستثمار. بالإضافة إلى النظرية الكينزية التي جاءت مع
أزمة الكساد العظيم \LR{The Great Depression} عام 1929.


\section{آليات ضبط الدورة الاقتصادية}
\label{sec:manage}
تحاول الحكومات ضبط الدورات الاقتصادية. يمكن للحكومة ضبط الدورة
الاقتصادية بضبط كميات الإنفاق الحكومي وضبط معدلات الضرائب.
فلإنهاء حالة الركود، تقوم الحكومة باستخدام أدوات لزيادة الإنفاق الحكومي
وتضئيل الضرائب، مما يساعد المنتجين والمستهلكين ويشجع الحركة الاقتصادية.
وبالعكس، تفرض الحكومة العكس لكي تمنع من
الانهاك الاقتصادي \LR{economy overheating}، حيث نلاحظ
التضخم الكبير، وأن الزيادة في الطلب تؤدي إلى ارتفاع الأسعار دون ارتفاع
كمية العرض.
تقوم البنوك المركزية أيضاً بمحاولة ضبط الدورات الاقتصادية،
بالتحكم	بمعدل الفائدة، وبيع وشراء السندات الحكومية، 
وتغيير كميات احتياطي البنوك \LR{bank reserves}.
حيث أنها ستقلل من معدل الفائدة عندما يكون الاقتصاد في حالة كساد،
وستزيد معدل الفائدة لتمنع الاقتصاد من الوصول إلى حالة قمة وبالتالي
زيادة مدّة حالة التوسع والانتعاش.

\section{المنفعة العملية من دراسة الدورات الاقتصادية (رأي شخصي)}
بات من الواضح أن الدورات الاقتصادية هي ظاهرة حاكمة وملازمة للنشاط الاقتصادي. فمن المستحيل تحقيق النمو الاقتصادي بشكل مستقيم دون وجود تقلبات موجية متمثلة بحالات
انتعاش وازدهار وأخرى متمثلة بحالات ركود وكساد. بفهمنا لأسباب نشوء هذه الدورات وآلية التقلبات التي تحدث فيها وما ينتج عنها، يمكننا بالتنبؤ بها واتخاذ القرارات المناسبة للحد
من حدّة هذه التقلبات. أو القيام بالنشاطات واتخاذ القرارات والأخذ بالاحتياطات المناسبة التي تسمح بالخروج من حالات الانحطاط وزيادة مدّة حالات الازدهار والتوسع.
فكما رأينا في الفقرة \ref{sec:manage} يمكن للحكومات والبنوك المركزية التأثير على تقلبات الدورة الاقتصادية، وكلما كانت هذه التأثيرات استباقية، كلما قلّت حدّة التقلبات
وزادت السيطرة على الدورة الاقتصادية.
أيضاً نرى أنه من المهم على مدراء الشركات معرفة وتحليل الوضع الحالي للاقتصاد؛ أي تحديد المرحلة الحالية لدورة الاقتصاد التي تحكم السوق. فهذا سيساعدهم لاتخاذ القرار
سواء لزيادة كميات العرض، أو سحبها من السوق، زيادة التوظيف في الشركة، تصغير حجم الشركة، زيادة الاستثمار وتحريك رأس المال الثابت، تجميد السيولة أو تحريكها.



