

\chapter{التقييم والنتائج}
يبيّن هذا الفصل النتائج التي حصلنا عليها بعد التنجيز والاختبار.
ويوضح أهم الملاحظات حول هذه النتائج.

\section{نتائج الـ \eng{OSE}}
إنّ أفضل نسبة صحّة تم تحقيقها على مجموعة المعطيات هذه هي $78.13\%$،
باستخدام $155$ سمة~\cite{vajjala2018}.
وللأسف لم يتم ذكر معايير أخرى لتقييم نتائجهم غير نسبة الصحّة.
نسبة الصحّة التي تم تحقيقها في هذا المشروع هي $80.83\%$.
وذلك باستخدام $30$ سمة فقط.
جميع النتائج المذكورة تخص مجموعة الاختبار المستخدمة.
حيث كما ذكرنا سابقاً، يوجد $567$ مثال تدريب.
تم فصل هذه العينات بنسبة $66\%$.
تم استخدام القسم الأكبر منها كمجموعة التدريب،
واستخدام القسم الأصغر كمجموعة الاختبار.

يبيّن الجدول~\ref{tbl:ose:metrics} قيم المعايير المختلفة المستخدمة لتقييم الأداء.
لمعرفة دلالاتها انظر إلى الفقرة~\ref{sec:metrics}.
إن مجمل النتائج جيد.
حيث نلاحظ أن جميع هذه القيم تتعدى الـ $0.7$.
وإن القيم المتوسطة تتعدى الـ $0.8$.
ونلاحظ أن المستوى الذي نقوم بتمييزه بالشكل الأفضل هو المستوى المبتدئ.
والمستوى الذي نقوم بتمييزه بالشكل الأسوء هو المستوى المتوسط،
وهذا متوقع كون المستوى المتوسط يقع بين المستويين المبتدئ والمتقدم مما يجعل احتمال الخطأ بتصنيفه أكبر.

يبيّن الجدول~\ref{tbl:ose:matrix} مصفوفة الحيرة للنتائج.
نلاحظ أن التمييز بين المستويين المبتدئ والمتقدم يتم بدون خطأ.
ونلاحظ أنه تم تصنيف عينات من المستوى المتوسط على أنها من المستوى المتقدم ومن المستوى المبتدئ بالتساوي (الرقم $10$ في الجدول).
الخطأ الأكبر في التصنيف هو تصنيف عينات من المستوى المتقدم على أنها من المستوى المتوسط (الرقم $13$ في الجدول).
قد يعود ذلك لكون الفارق بين المستويين المتوسط والمبتدئ أكبر من الفارق بين المستويين المتوسط والمتقدم.

\begin{table}[htb]
	\centering
	{
		\setlength{\tabcolsep}{0.5em} % for the horizontal padding
		\renewcommand{\arraystretch}{1.4}% for the vertical padding
		\selectlanguage{english}
		
		\begin{tabular}{|c|c|c|c|}
			\hline
			
			Level &
			Precision &
			Recall &
			F-Score \\
			\hline
			
			Elementary &
			0.865 &
			0.941 &
			0.901 \\
			\hline
			
			Intermediate &
			0.742 &
			0.710 &
			0.726 \\
			\hline
			
			Advanced &
			0.811 &
			0.768 &
			0.789 \\
			\hline
			
			Average &
			0.806 &
			0.808 &
			0.806 \\
			\hline
			
		\end{tabular}
	}
	\caption{%
		معايير تقييم الأداء لمعطيات الـ \eng{OSE}.
	}
	\label{tbl:ose:metrics}
\end{table}

\begin{table}[htb]
	\centering
	{
		\setlength{\tabcolsep}{0.5em} % for the horizontal padding
		\renewcommand{\arraystretch}{1.4}% for the vertical padding
		\selectlanguage{english}
		
		\begin{tabular}{|c|c|c|c|}
			\hline
			
			\backslashbox{actual}{predicted} &
			Elementary &
			Intermediate &
			Advanced \\
			\hline
			
			Elementary &
			64 &
			4 &
			0 \\
			\hline
			
			Intermediate &
			10 &
			49 &
			10 \\
			\hline
			
			Advanced &
			0 &
			13 &
			43 \\
			\hline
			
			
		\end{tabular}
	}
	\caption{%
		مصفوفة الحيرة لمعطيات الـ \eng{OSE}.
	}
	\label{tbl:ose:matrix}
\end{table}




