

\chapter{الاختبارات والنتائج}
todo

\section{نتائج الـ \eng{OSE}}
todo
إنّ أفضل نسبة صحّة تم تحقيقها على مجموعة المعطيات هذه هي $78.13\%$،
باستخدام $155$ ميزة~\cite{vajjala2018}.
وللأسف لم يتم ذكر معايير أخرى لتقييم نتائجهم غير نسبة الصحّة.
نسبة الصحّة التي تم تحقيقها في هذا المشروع هي $80.83\%$.
يبيّن الجدول~\ref{tbl:ose:metrics} قيم المعايير المختلفة المستخدمة لتقييم الأداء.
لمعرفة دلالاتها انظر إلى الفقرة~\ref{sec:metrics}.
كما يبيّن الجدول~\ref{tbl:ose:matrix} مصفوفة الحيرة.

جميع النتائج المذكورة تخص مجموعة الاختبار المستخدمة.
حيث كما ذكرنا سابقاً، يوجد $567$ مثال تدريبي.
تم فصل هذه العينات بنسبة $66\%$.
تم استخدام القسم الأكبر منها كمعطيات التدريب،
واستخدام القسم الأصغر كمعطيات الاختبار.

\begin{table}[htb]
	\centering
	{
		\setlength{\tabcolsep}{0.5em} % for the horizontal padding
		\renewcommand{\arraystretch}{1.4}% for the vertical padding
		\selectlanguage{english}
		
		\begin{tabular}{|c|c|c|c|}
			\hline
			
			Level &
			Precision &
			Recall &
			F-Score \\
			\hline
			
			Elementary &
			0.865 &
			0.941 &
			0.901 \\
			\hline
			
			Intermediate &
			0.742 &
			0.710 &
			0.726 \\
			\hline
			
			Advanced &
			0.811 &
			0.768 &
			0.789 \\
			\hline
			
			Average &
			0.806 &
			0.808 &
			0.806 \\
			\hline
			
		\end{tabular}
	}
	\caption{%
		معايير تقييم الأداء لمعطيات الـ \eng{OSE}.
	}
	\label{tbl:ose:metrics}
\end{table}

\begin{table}[htb]
	\centering
	{
		\setlength{\tabcolsep}{0.5em} % for the horizontal padding
		\renewcommand{\arraystretch}{1.4}% for the vertical padding
		\selectlanguage{english}
		
		\begin{tabular}{|c|c|c|c|}
			\hline
			
			\backslashbox{actual}{predicted} &
			Elementary &
			Intermediate &
			Advanced \\
			\hline
			
			Elementary &
			64 &
			4 &
			0 \\
			\hline
			
			Intermediate &
			10 &
			49 &
			10 \\
			\hline
			
			Advanced &
			0 &
			13 &
			43 \\
			\hline
			
			
		\end{tabular}
	}
	\caption{%
		مصفوفة الحيرة لمعطيات الـ \eng{OSE}.
	}
	\label{tbl:ose:matrix}
\end{table}




